\documentclass[12pt,a4paper,twoside]{book}


\usepackage[utf8]{inputenc}
\usepackage[a4paper,inner=3.5cm,outer=2.5cm]{geometry}

\usepackage[titletoc,title,toc,page]{appendix}
\usepackage{verbatim}
\usepackage{placeins}
\usepackage{listings}
\usepackage{hyperref}
\usepackage[italian]{babel}
\usepackage{tikz}
\usepackage{parskip}

\usepackage{graphicx}
\usepackage{blindtext}
\usepackage{chngcntr}
\counterwithin{table}{chapter}

\usepackage{newlfont}
\usepackage{fancyhdr}
\usepackage{indentfirst}
\usepackage[utf8]{inputenc}
\usepackage{float}
\usepackage{hyperref}
\usepackage{soul}
\usepackage[font=footnotesize,labelfont=bf]{caption}

\usepackage{multirow}
\usepackage{hyphenat}
\hyphenation{mate-mati-ca recu-perare}

\usepackage{lscape} 

\usepackage{natbib}
\bibliographystyle{alpha}
\setcitestyle{super,open={[},close={]}}

\newcommand{\rom}[1]{\uppercase\expandafter{\romannumeral #1\relax}}

\usepackage{pdfpages}

\begin{document}
% Per spostare i vari elementi più su o più giù gioca con i valori di vspace che ci sono tra uno e l'altro
\pagestyle{empty}
\newgeometry{left=2cm, right=2cm}
\begin{titlepage}
\begin{center}
    {{\Large{\textsc{Alma Mater Studiorum $\cdot$ Università di Bologna}}}}
    \rule[0.1cm]{\textwidth}{0.1mm}
    \rule[0.5cm]{\textwidth}{0.6mm}\\
    {\small{\bf SCUOLA DI SCIENZE\\
    Corso di Laurea in Informatica}}
\end{center}

\vspace{25mm}

\begin{center}
    {\LARGE{\bf TITOLO DELLA }}\\
    \vspace{3mm}
    {\LARGE{\bf TESI ANCHE SU}}\\
    \vspace{3mm}
    {\LARGE{\bf PIÙ RIGHE}}\\
\end{center}

\vspace{60mm}
\par
\noindent
\begin{minipage}[t]{0.04\textwidth}
~
\end{minipage}
\begin{minipage}[t]{0.4\textwidth}
{\large{\bf Relatore:\\
Chiar.mo Prof.\\
RELATORE}}
\end{minipage}
\hfill
\begin{minipage}[t]{0.4\textwidth}\raggedleft
{\large{\bf Presentata da:\\
TU}}
\end{minipage}
\begin{minipage}[t]{0.04\textwidth}
~
\end{minipage}

\vspace{30mm}

\begin{center}
    {\large{\bf \rom{1} Sessione\\
    Anno Accademico 2022/2023 }}
\end{center}
\end{titlepage}

\restoregeometry
\newpage
\begin{center}
    (DA FARE ALLA FINE)\\
    5 parole chiave per caratterizzare il contenuto della dissertazione:\\ (se non ti piacciono così sparse puoi anche semplicamente scriverle su una riga sola)
\end{center}

% https://tex.stackexchange.com/questions/26538/words-scattered-randomly-in-on-coverpage
\begin{tikzpicture}[overlay,remember picture,shift=(current page.center)]
\pgfmathsetseed{3}


\foreach [count=\count] \word in {Parola 1, parola 2, parola 3, parola 4, parola 5} {
\node [
    xshift={(mod(\count,3)-1)*(\paperwidth/4)},
    yshift={(mod(\count,7)-3)*(\paperwidth/6)},
    xshift=rand*4cm,
    yshift=rand*2cm,
    % rotate=rand*35,
    % opacity=rnd*0.5+0.125,
    font=\large] {\word};
}
\end{tikzpicture}
\newpage

\topmargin=6.5cm
\begin{flushright}
\emph{
\LARGE{La dedica}\\\vspace{2mm}
\LARGE{anche quella se vuoi}\\\vspace{3mm} 
\LARGE{su più righe} 
}
\end{flushright}
\newpage~\newpage
\pagenumbering{gobble}
\chapter*{Abstract}
Abstract qui (ti consiglio di farlo alla fine)

\topmargin=-1cm
\tableofcontents
\thispagestyle{empty}
\listoftables
\thispagestyle{empty}
\listoffigures
\thispagestyle{empty}
\newpage~\newpage


\pagenumbering{arabic}
\setcounter{chapter}{-1}
\raggedbottom
\chapter{INTRODUZIONE} \label{chap:intro}
\pagestyle{plain}
\setcounter{page}{1}

(Io l'introduzione l'ho scritta alla fine)\\
bla bla bla
\section{elenchi}
\subsection{Elenchi puntati}
\begin{itemize}
    \item bla
    \begin{itemize}
        \item sub-bla
    \end{itemize}
    \item bla
\end{itemize}

\subsection{Elenchi numerati}
\begin{enumerate}
    \item bla1
    \begin{enumerate}
        \item sub bla 1
        \item sub bla 2
    \end{enumerate}
    \item bla 2
\end{enumerate}

\subsection{Mix}
\begin{itemize}
    \item bla
    \begin{enumerate}
        \item sub bla 1
    \end{enumerate}
    \item bla
\end{itemize}

\begin{enumerate}
    \item bla 1
    \begin{itemize}
        \item sub bla
    \end{itemize}
    \item bla 2
\end{enumerate}

\section{Font}
\textbf{bla bla bla}\\
\textit{Ancora bla bla bla}\\
\texttt{bla bla ma in un'altra riga}

\subsection{Sottosezione 1}
Grazie al package parskip se vai a capo nel .tex lasciando una riga

ti mette un po' di spazio anche nel pdf.\\
Attenzione però che ogni tanto questa feature fa lasciare troppo spazio tra testo e immagini / tabelle, se capita prova a togliere un po' di righe vuote.
\subsection{Sottosezione 2}
I capitoli iniziano sempre in una pagina dispari, quindi a volte vedrai delle pagine bianche tra uno e l'altro
\subsubsection{Sottosottosezione 1} \label{subsub:bla}
bla bla bla

\chapter{Dopo l'introduzione}
qua scrivi qualcosa
\section{Immagini}
Quando fai begin figure, ricordati di mettere tra quadre un modificatore di posizione: H significa esattamente nel punto dove si trova l'immagine nel file .tex e ti consiglio di usare quello, se no ci sono ad esempio t (top) e b (bottom).

\begin{figure}[H]
    \centering
    % Se metti solo una delle due dimensioni, l'altra scala in automatico
    \includegraphics[height = 6cm, width=8cm]{img/frog.jpg}
    \caption{Caption (questo viene scritto nell'indice delle figure)}
    % La label ci vuole sempre e te la inventi tu: serve per riferirsi alle immagini successivamente
    \label{fig:frog}
\end{figure}

\section{tabelle}
\subsection{Tabella semplice}
Anche qui nota H tra quadre, la caption e la label

\begin{table}[H]
    \centering
    \begin{tabular}{|c|c|}
    \hline
        \textbf{Pratiche agili} & \textbf{Studenti}  \\ \hline
        Sprint planning & 73  \\ \hline
        Pair programming & 73  \\ \hline
        Retrospettiva & 48  \\ \hline
    \end{tabular}
    \caption{Tabella semplice (anche questo scritto nell'indice delle tabelle)}
    \label{tab:simple}
\end{table}

\subsection{tabelle avanzate}
Con multirow (e multicolumn che però serve meno) puoi fare righe (colonne) più grandi del normale.\\
\begin{table}[H]
    \centering
    \begin{tabular}{|c|c|c|c|c|}
    \hline
        \textbf{Team} & \textbf{LoC verificate} & \textbf{LoC sviluppatori} & \textbf{Ore sviluppatori} & \textbf{LoC/h}  \\ \hline
        \multirow{2}*{1} & 1148& m: 888& m: 40& 22\cr & Diff: -1852& $\sigma$: 371& $\sigma$: 27& \\ \hline
        \multirow{2}*{2}&1858& m: 1404& m: 65& 22\cr &  Diff: -448& $\sigma$: 1222& $\sigma$: 78& \\ \hline
        \multirow{2}*{3}&1640& m: 1400& m: 96& 15\cr &  Diff: -2810& $\sigma$: 1417& $\sigma$: 41& \\ \hline
    \end{tabular}
    \caption{CAPTION}
    \label{tab:avanz}
\end{table}

\subsubsection{Tabelle girate}
Se usi landscape la tabella viene girata (nel caso dovessi inserirne una molto grande)
\begin{landscape}
\begin{table}[H]
    \centering
    \begin{tabular}{|c|c|}
    \hline
        \textbf{Pratiche agili} & \textbf{Studenti}  \\ \hline
        Sprint planning & 73  \\ \hline
        Pair programming & 73  \\ \hline
        Retrospettiva & 48  \\ \hline
    \end{tabular}
    \caption{Tabella girata}
    \label{tab:girata}
\end{table}

\end{landscape}


\chapter{Altri comandi}
bla bla
\section{Math mode}
Per inserire simboli matematici (e lettere greche) serve la math mode:

Usando il simbolo del dollaro hai la math mode inline: $5 \times \alpha = 3\lambda$

Altrimenti hai quella con le barre e le quadre \[ \frac{\sum_6^i 3i\theta}{12k^2\times 7}\]

Infine hai quelle con begin equation (che vengono numerate):
\begin{equation}
    \frac{1}{2}\times A_{bcd}\times E^{fgh}
\end{equation}

Anche le equazioni possono avere label.
\section{url e footnote}
per mettere un link usa url: \url{wikipedia.it}

per fare note a piè di pagina usa footnote\footnote{Tipo questa}
\section{verbatim}
Se ti serve scrivere codice o qualcosa per cui ti serve una formattazione specifica usa verbatim:
\begin{verbatim}
    Qui puoi scrivere

    come      vuoi
    e viene tutto

scritto
                    monospaziato
\end{verbatim}
\section{riferimenti}
Come detto prima le label servono per riferirsi ad altre parti del testo citate precendentemente.\\
Ti consiglio di metterle sempre almeno a figure. immagini e capitoli.

Per riferirti a qualcosa basta fare ref seguito dal nome della label, ad esempio ``vedi capitolo \ref{chap:intro}''.\\In questo modo dal pdf cliccando sulla reference, ti porta direttamente al punto giusto.

\section{citazioni}
Per citare si usa cite seguito dal nome dell'articolo nel file.bib, ad esempio ``come visto nell'articolo di tizio\cite{greenwade93}''.

Se non ti piace lo stile di citazione puoi modificarlo sopra dove scrivo usepackage natbib, ma quello impostato attualmente dovrebbe andare bene.



\renewcommand{\bibsection}{}
\chapter*{Riferimenti bibliografici}
\bibliography{refs}
\newpage

\renewcommand{\appendixtocname}{Appendici}
\renewcommand{\appendixpagename}{Appendici}
% \csname @openrightfalse\endcsname
\pagenumbering{gobble}
\begin{appendices}
\chapter{Appendice 1}
\label{Appendice:A}
Probabilmente ci sono un sacco di package non utilizzati ma così funziona tutto quindi non ho indagato oltre.

Inoltre su internet c'è un sacco di documentazione se ti servisse.
\chapter{Appendice B}
\label{Appendice:B}
Appendice B se serve

\chapter{Embed di interi PDF}
\label{Appendice:C}
Se ti serve puoi fare embed di PDF interi con pdfpage, scegliendo anche le pagine (o mettendo - se le vuoi tutte):

\includepdf[pages=1]{pdf/Requisiti.pdf}
\end{appendices}

\newpage~\newpage
\chapter*{Ringraziamenti}
Grazie a tutti
\end{document}